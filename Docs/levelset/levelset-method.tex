
\documentclass[12pt]{report}


%%% --- Packages Required --- %%%
\usepackage{times}             % sets font (times new roman)
\usepackage{geometry}          % allows for manipulation of document geometry
\usepackage{graphicx}          % allows images to be added
\usepackage{amsmath}           % enables equations
\usepackage{siunitx}
\usepackage[version=3]{mhchem} % enables chemical equations
\usepackage{bm}                % enables bold math symboles
\usepackage{subfiles}          % enables subfiles (load this package last)
\usepackage{titlesec}          % customisation of chapters
\usepackage{enumitem}          % customisation of  bullet points
\usepackage{sectsty}           % customisation of sections styles
\usepackage{fancyhdr}          % generates fancy headers
\usepackage{hyperref}          % enables hyperlinks
\usepackage{caption}
\usepackage{subcaption}
\usepackage{xargs}              % Use more than one parameter in command
\usepackage[colorinlistoftodos,prependcaption,textsize=tiny]{todonotes}
%\usepackage[round, authoryear]{natbib} % for referencing

\setlength{\parindent}{0pt}

\begin{document}

\section*{The reinitialisation algorithm:}

if it is the first step (ie start of simulation),  initialise a gfield with a smooth function:

\[\phi_{i,j,k} = d^2 \]

where $d$ is the distance in the y direction from the desired surface position and $\phi$ is the value of the level set, where $\phi < 0$ in the reactants, $\phi = 0$ at the surface and $\phi > 0$ in the products.

To reintialise the levelset the following equation is used:

\[ \frac{\partial \phi}{\partial \tau} = S (\phi^0) (\lvert \nabla \phi \rvert - 1)  \]

where $\phi^0$ is $\phi$ before reinitialisation (ie $S(\phi^0)$ does not change thoughout the reinitialisation procedure). $\tau$ is an artifical timestep to propagate the reinitialisation.

To renitialise the levelset, the following algorithm is used:

set $S(\phi^0)$:
\[S(\phi^0) = \frac{\phi_{i,j,k}}{\sqrt{{\phi_{i,j,k}^2 + dx^2}}} \]

for ($N=0, N<nsteps, N++$):
\begin{itemize}
\item calculate $\lvert \nabla \phi \rvert$
\item calculate new $\phi_{i,j,k}$ (ie $\phi^{n+1}_{i,j,k}$)
\end{itemize}


\subsection*{calculating $\lvert \nabla \phi \rvert$}
Example for 2D:

\[\Delta^-_x = \frac{\phi_{i,j,k} - \phi_{i-1,j,k}}{dx}  \]
\[\Delta^+_x = \frac{\phi_{i+1,j,k} - \phi_{i,j,k}}{dx}  \]
\[a^- = \min(\Delta^-_x,0)  \]
\[a^+ = \max(\Delta^-_x,0)  \]
\[b^- = \min(\Delta^+_x,0)  \]
\[b^+ = \max(\Delta^+_x,0)  \]
\[\Delta^-_y = \frac{\phi_{i,j,k} - \phi_{i,j-1,k}}{dy}  \]
\[\Delta^+_y = \frac{\phi_{i,j+1,k} - \phi_{i,j,k}}{dy}  \]
\[c^- = \min(\Delta^-_y,0)  \]
\[c^+ = \max(\Delta^-_y,0)  \]
\[d^- = \min(\Delta^+_y,0)  \]
\[d^+ = \max(\Delta^+_y,0)  \]

\[ \lvert \nabla \phi_{i,j,k} \rvert =
\begin{cases}
  \sqrt{\max((a^+)^2,(b^-)^2) + \max((c^+)^2,(d^-)^2)} & \text{if } S(\phi^0_{i,j,k}) > 0 \\
  \sqrt{\max((a^-)^2,(b^+)^2) + \max((c^-)^2,(d^+)^2)} & \text{else}
\end{cases}
\]

\subsection*{calculating $\phi^{n+1}_{i,j,k}$}

\[ \phi^{n+1}_{i,j,k} = \phi_{i,j,k} - \tau S(\phi^0_{i,j,k}) (\lvert \nabla \phi_{i,j,k} \rvert - 1) \]

where $\tau = 0.5 dx$ in 2D.

\end{document}
